\documentclass{article}

\title{Autocorrelation in Weather: Results}
\author{Lucy Goodyear}
\date{21st October 2019}

\usepackage{Sweave}
\begin{document}

\maketitle
\Sconcordance{concordance:TAutoCorr.tex:TAutoCorr.Rnw:%
1 6 1 1 0 16 1}


\section{Introduction}

The code below calculates the correlation coefficient for temperature between successive years in West Keyes, Florida. It then calculates the correlation coefficient for 10,000 random permutations of the temperature in order to compare significance. It is expecting that the number of permutations with a significant correlation will be non-zero (due to random correlartion) but very low.

\section{Code}

\section{Results}

The precentage of correlation coefficients produced from random temperature permuations that were more significant than the correlation coeeficient for successive years, when run for the first time, was 0.06\% (or 6 out of the 10,000 samples). This agrees with the prediction at the start of this paper.


\end{document}
